\begin{thebibliography}{99}

%----------------------------CAPITULO 1------------------------%
%MARCO DE DESARROLLO
\bibitem{equipo_cajal} Research Team. Neural Rehabilitation Group CSIC. Disponible en: \url{http://www.neuralrehabilitation.org/en/?page_id=15} [Último acceso 1 de abril, 2020].


%MEDULA ESPINAL
\bibitem{anatomia_medula_1} Guertin \& A., P., 2012. Central Pattern Generator for Locomotion: Anatomical, Physiological, and Pathophysiological Considerations. Frontiers. Disponible en: \url{https://www.frontiersin.org/articles/10.3389/fneur.2012.00183/full} [Último acceso 9 de abril, 2020].

\bibitem{anatomia_medula_2} Lab 2 Spinal Cord Gross Anatomy. Disponible en: \url{http://vanat.cvm.umn.edu/neurLab2/SpCdGross.html} [Último acceso 9 de abril, 2020].

\bibitem{columna_vertebral} Columna Vertebral. de Columna Vertebral: Características y Partes. Disponible en: \url{https://significado.com/columna-vertebral/} [Último acceso 9 de abril, 2020].

\bibitem{seccion_medula} Wendy, Human Anatomy and Physiology Lab (BSB 141). Lumen. Disponible en: \url{https://courses.lumenlearning.com/ap1x94x1/chapter/the-spinal-cord/} [Último acceso 9 de abril, 2020].

\bibitem{dermatomas} Medspine, C., 2017. Dermatomas y miotomas. www.medspine.es.  Disponible en: \url{https://www.medspine.es/dermatomas/} [Último acceso 9 de abril, 2020].

\bibitem{dermatomas_puntos} Poynton, A.R. et al., 2019. Fig. 1 Location of the key sensory points for each dermatome... ResearchGate. Disponible en: \url{https://www.researchgate.net/figure/Location-of-the-key-sensory-points-for-each-dermatome-reproduced-with-the-permission-of_fig1_13836050} [Último acceso 9 de abril, 2020].


%LESIONES MEDULARES
\bibitem{sci_clasificacion} Maynard, F.M. \& Bracken, Michael \& Creasey, Graham \& Ditunno, J.F. \& Donovan, William \& Ducker, Thomas \& Garber, Susan \& Marino, Ralph \& Stover, Samuel \& Tator, Charles \& Waters, Robert \& Wilberger, Jack \& Young, Wise. (1997). International Standards for Neurological and Functional Classification of Spinal Cord Injury. American Spinal Injury Association. Spinal cord. 35. 266-74. 10.1038/sj.sc.3100432.  Disponible en: \url{https://www.researchgate.net/publication/14059061_International_Standards_for_Neurological_and_Functional_Classification_of_Spinal_Cord_Injury_American_Spinal_Injury_Association} [Último acceso 9 de abril, 2020].

\bibitem{causas_sci} Mayo Clinic, 2019. Spinal cord injury. Disponible en: \url{https://www.mayoclinic.org/diseases-conditions/spinal-cord-injury/symptoms-causes/syc-20377890} [Último acceso 25 de agosto, 2020].

\bibitem{ASIA} The premier North American organization in the field of Spinal Cord Injury Care, Education, and Research. American Spinal Injury Association. Disponible en: \url{https://asia-spinalinjury.org/} [Último acceso 7 de mayo, 2020].

\bibitem{estandar_asia} Menéndez, S.R., 2015. Lesión medular: Clasificación ASIA. Neurofuncion.  Disponible en: \url{https://neurofuncion.com/lesion-medular-clasificacion-asia/} [Último acceso 9 de abril, 2020].

\bibitem{examen_fim} FUNCTIONAL INDEPENDENCE MEASURE - ADL measures for people post-stroke. Google Sites. Disponible en: \url{https://sites.google.com/site/movementincontext6/the-assessments/functional-independence-measure} [Último acceso 9 de abril, 2020].

\bibitem{tesis_antonio} Espinosa, Ama and Antonio del. “Hybrid walking therapy with fatigue management for spinal cord injured individuals.” (2013). Disponible en: \url{https://e-archivo.uc3m.es/handle/10016/18294} [Último acceso 9 de abril, 2020].


%REHABILITACION
\bibitem{rehabilitacion} Pillastrini, P. et al., 2007. Evaluation of an occupational therapy program for patients with spinal cord injury. Nature News. Disponible en: \url{https://www.nature.com/articles/3102072} [Último acceso 9 de abril, 2020].

\bibitem{etapas_rehabilitacion} Argy Stampas, 2013. Recovering from Spinal Cord Injury: Treatment Stages. Burke Rehabilitation Hospital. Disponible en: \url{https://www.burke.org/media/news/2013/02/recovering-from-spinal-cord-injury-treatment/63} [Último acceso 9 de abril, 2020].

\bibitem{rehabilitacion_caminar} Hubli, Michèle \& Dietz, Volker. (2013). The physiological basis of neurorehabilitation - Locomotor training after spinal cord injury. Journal of neuroengineering and rehabilitation. 10. 5. 10.1186/1743-0003-10-5.  Disponible en: \url{https://jneuroengrehab.biomedcentral.com/articles/10.1186/1743-0003-10-5} [Último acceso 9 de abril, 2020].

\bibitem{ciclo_marcha} Protokinetics Team \& Protokinetics Admin, 2019. Phases of the Gait Cycle: Gait Analysis " ProtoKinetics. ProtoKinetics. Disponible en: \url{https://www.protokinetics.com/understanding-phases-of-the-gait-cycle/} [Último acceso 2 de mayo, 2020].

\bibitem{recovery_locomotion} Rossignol1, S. \& Frigon1, A., Recovery of Locomotion After Spinal Cord Injury: Some Facts and Mechanisms. Annual Reviews.  Disponible en: \url{https://www.annualreviews.org/doi/abs/10.1146/annurev-neuro-061010-113746} [Último acceso 9 de abril, 2020].

%exoesqueletos
\bibitem{exoesqueletos} Ladan Najafi \& Donna Cowan, Handbook of Electronic Assistive Technology. ScienceDirect. Disponible en: \url{https://www.sciencedirect.com/book/9780128124871/handbook-of-electronic-assistive-technology} [Último acceso 4 de mayo, 2020].

\bibitem{comparacion_exoesqueletos} Chiaradia, Domenico \& Xiloyannis, Michele \& Solazzi, Massimiliano \& Masia, Lorenzo \& Frisoli, Antonio. (2018). Comparison of a Soft Exosuit and a Rigid Exoskeleton in an Assistive Task.  Disponible en: \url{https://www.researchgate.net/publication/327345060_Comparison_of_a_Soft_Exosuit_and_a_Rigid_Exoskeleton_in_an_Assistive_Task} [Último acceso 4 de mayo, 2020].

\bibitem{estudio_exoesqueletos} Sanchez-Villamañan, M.D.C. et al., 2019. Compliant lower limb exoskeletons: a comprehensive review on mechanical design principles. Journal of neuroengineering and rehabilitation. Disponible en: \url{https://www.ncbi.nlm.nih.gov/pmc/articles/PMC6506961/} [Último acceso 4 de mayo, 2020].

\bibitem{exoesqueleto_rigido}  Knowledge shared by technology developers and scientists. Wevolver. Disponible en: \url{https://www.wevolver.com/wevolver.staff/vlexo/} [Último acceso 4 de mayo, 2020].

\bibitem{exoesqueleto_flexible} Bae, Jaehyun. (2018). A lightweight and efficient portable soft exosuit for paretic ankle assistance in walking after stroke.  Disponible en: \url{https://www.researchgate.net/publication/326262815_A_lightweight_and_efficient_portable_soft_exosuit_for_paretic_ankle_assistance_in_walking_after_stroke} [Último acceso 4 de mayo, 2020].

\bibitem{actuadores_exoesqueletos} Manna, Soumya \& Dubey, Venketesh. (2018). Comparative study of actuation systems for portable upper limb exoskeletons. Medical Engineering \& Physics. 60. 10.1016/j.medengphy.2018.07.017.  Disponible en: \url{https://www.researchgate.net/publication/327085558_Comparative_study_of_actuation_systems_for_portable_upper_limb_exoskeletons} [Último acceso 4 de mayo, 2020].

\bibitem{robotica_rehabilitacion} Hidler, J. \& Sainburg, R., 2011. Role of Robotics in Neurorehabilitation. Topics in spinal cord injury rehabilitation. Disponible en: \url{https://www.ncbi.nlm.nih.gov/pmc/articles/PMC3157701/} [Último acceso 3 de mayo, 2020].

\bibitem{assist_as_need} L. L. Cai, A. J. Fong, Yongqiang Liang, J. Burdick and V. R. Edgerton, "Assist-as-needed training paradigms for robotic rehabilitation of spinal cord injuries," Proceedings 2006 IEEE International Conference on Robotics and Automation, 2006. ICRA 2006., Orlando, FL, 2006, pp. 3504-3511. Disponible en: \url{https://ieeexplore.ieee.org/abstract/document/1642237} [Último acceso 9 de abril, 2020].

\bibitem{ventajas_desventajas_exoesqueletos} Gorgey, A.S., 2018. Robotic exoskeletons: The current pros and cons. World journal of orthopedics. Disponible en: \url{https://www.ncbi.nlm.nih.gov/pmc/articles/PMC6153133/} [Último acceso 4 de mayo, 2020].

\bibitem{lokomat} Bryant, P. et al., Lokomat®. Hocoma. Disponible en: \url{https://www.hocoma.com/solutions/lokomat/?variation=LokomatPro#product} [Último acceso 9 de abril, 2020].

\bibitem{lokomat_imagen} Gnocchi, F.D., Lokomat. EASTIN. Disponible en: \url{http://www.eastin.eu/es-es/searches/products/detail/database-rehadat/id-M_26959} [Último acceso 9 de abril, 2020].

\bibitem{control_exoesqueleto} Anam, K. \& Al-Jumaily, A.A., 2012. Active Exoskeleton Control Systems: State of the Art. Procedia Engineering. Disponible en: \url{https://www.sciencedirect.com/science/article/pii/S1877705812026732} [Último acceso 6 de mayo, 2020].

%electroestimulacion
\bibitem{electroestimulacion} Ho, C.H. et al., 2014. Functional electrical stimulation and spinal cord injury. Physical medicine and rehabilitation clinics of North America. Disponible en: \url{https://www.ncbi.nlm.nih.gov/pmc/articles/PMC4519233/} [Último acceso 9 de abril, 2020].

\bibitem{electroestimulacion2} Hamid, S. \& Hayek, R., 2008. Role of electrical stimulation for rehabilitation and regeneration after spinal cord injury: an overview. European spine journal : official publication of the European Spine Society, the European Spinal Deformity Society, and the European Section of the Cervical Spine Research Society. Disponible en: \url{https://www.ncbi.nlm.nih.gov/pmc/articles/PMC2527422/} [Último acceso 3 de mayo, 2020].

\bibitem{sanitas} ¿Para qué sirve la electroestimulación? Sanitas. Disponible en: \url{https://www.sanitas.es/sanitas/seguros/es/particulares/biblioteca-de-salud/estetica/electroestimulacion.html} [Último acceso 1 de junio, 2020].

\bibitem{tipos_electrodos} Bajd, Tadej \& Munih, Marko. (2010). Basic functional electrical stimulation (FES) of extremities: An engineer's view. Technology and health care : official journal of the European Society for Engineering and Medicine. 18. 361-9. 10.3233/THC-2010-0588.   Disponible en: \url{https://www.researchgate.net/publication/49731928_Basic_functional_electrical_stimulation_FES_of_extremities_An_engineer's_view} [Último acceso 3 de mayo, 2020].

\bibitem{electrodo_superficial} Admin, I.R.F., 2019. Neuromodulación del tibial posterior: una opción de tratamiento en la vejiga neurógena. IRF La Salle - Centro de Rehabilitación Aravaca - Madrid. Disponible en: \url{https://www.irflasalle.es/neuromodulacion-del-tibial-posterior-una-opcion-tratamiento-la-vejiga-neurogena/} [Último acceso 9 de abril, 2020].

\bibitem{electrodo_percutaneo} Estimulación eléctrica percutánea de nervios periféricos (PENS). Medicina del Dolor. Disponible en: \url{https://medicinadeldolor.es/tratamientos/tecnicas-de-neuromodulacion-espinal-y-periferica/estimulacion-electrica-percutanea-de-nervios-perifericos-pens/} [Último acceso 9 de abril, 2020].

\bibitem{FES} Casco, S. \& Fuster, I. \& Galeano, Ramona \& Moreno, Juan \& Pons, J.L. \& Brunetti, Fernando. (2017). Towards an ankle neuroprosthesis for hybrid robotics: Concepts and current sources for functional electrical stimulation. IEEE ... International Conference on Rehabilitation Robotics : [proceedings]. 2017. 1660-1665. 10.1109/ICORR.2017.8009486.  Disponible en: \url{https://www.researchgate.net/publication/319119132_Towards_an_ankle_neuroprosthesis_for_hybrid_robotics_Concepts_and_current_sources_for_functional_electrical_stimulation} [Último acceso 2 de mayo, 2020].

\bibitem{parametros_FES} E. Krueger-Beck, E. M. Scheeren, G. N. Nogueira-Neto, V. L. S. N. Button and P. Nohama, "Optimal FES parameters based on mechanomyographic efficiency index," 2010 Annual International Conference of the IEEE Engineering in Medicine and Biology, Buenos Aires, 2010, pp. 1378-1381.  Disponible en: \url{https://ieeexplore.ieee.org/document/5626735} [Último acceso 3 de mayo, 2020].

\bibitem{tipos_repeticion}Cometti, C., Babault, N. \& Deley, G., 2016. Effects of Constant and Doublet Frequency Electrical Stimulation Patterns on Force Production of Knee Extensor Muscles. PloS one.  Disponible en: \url{https://www.ncbi.nlm.nih.gov/pmc/articles/PMC4864221/ } [Último acceso 2 de mayo, 2020].

\bibitem{catch_like} Fortuna, R., Vaz, M.A. \& Herzog, W., 2011. Catchlike property in human adductor pollicis muscle. Journal of Electromyography and Kinesiology.  Disponible en: \url{https://www.sciencedirect.com/science/article/pii/S1050641111001532} [Último acceso 3 de mayo, 2020].

\bibitem{ventajas_FES} Rushton, D.N., 2002. Functional Electrical Stimulation and rehabilitation-an hypothesis. Medical Engineering \& Physics. Disponible en: \url{https://www.sciencedirect.com/science/article/pii/S1350453302000401} [Último acceso 3 de mayo, 2020].

\bibitem{limitaciones_fes} Popovic, M., Curt, A., Keller, T. et al. Functional electrical stimulation for grasping and walking: indications and limitations. Disponible en: \url{https://doi.org/10.1038/sj.sc.3101191} [Último acceso 9 de abril, 2020].

\bibitem{pie_caido} Pie caído. Mayo Clinic. Disponible en: \url{https://www.mayoclinic.org/es-es/diseases-conditions/foot-drop/symptoms-causes/syc-20372628} [Último acceso 1 de abril, 2020].

\bibitem{estimulador_odstock} ODFS® Pace Kit: Odstock Medical Ltd (OML). ODFS® Pace Kit | Odstock Medical Ltd (OML). Disponible en: \url{https://www.odstockmedical.com/sites/default/files/oml_fes_products_and_services_2011_1.pdf} [Último acceso 2 de mayo, 2020].

\bibitem{control_FES} Ambrosini, E. et al., 2015. Control system for neuro-prostheses integrating induced and volitional effort. IFAC-PapersOnLine. Disponible en: \url{https://www.sciencedirect.com/science/article/pii/S2405896315020510} [Último acceso 6 de mayo, 2020].

\bibitem{estado_arte_FES} Gil-Castillo, J., Alnajjar, F., Koutsou, A. et al. Advances in neuroprosthetic management of foot drop: a review. J NeuroEngineering Rehabil 17, 46 (2020). Disponible en: \url{https://doi.org/10.1186/s12984-020-00668-4} [Último acceso 31 de agosto, 2020].

%protesis hibridas
\bibitem{protesis_hibridas} Jacob Rosen \& Peter Walker Ferguson, Wearable Robotics. ScienceDirect. Disponible en: \url{https://www.sciencedirect.com/book/9780128146590/wearable-robotics} [Último acceso 25 de agosto, 2020].

\bibitem{FES_exoesqueleto_motriz} IEEE, An Approach for the Cooperative Control of FES With a Powered Exoskeleton During Level Walking for Persons With Paraplegia - IEEE Journals \&amp; Magazine. Disponible en: \url{https://ieeexplore.ieee.org/document/7093196} [Último acceso 25 de agosto, 2020].

\bibitem{requerimientos_ambulacion_1} Lapointe, R. et al., 2001. Functional community ambulation requirements in incomplete spinal cord injured subjects. Nature News. Disponible en: \url{https://www.nature.com/articles/3101167} [Último acceso 25 de agosto, 2020].

\bibitem{requerimientos_ambulacion_2} Robinett, C.S. \&amp; Vondran, M.A., 1988. Functional Ambulation Velocity and Distance Requirements in Rural and Urban Communities: A Clinical Report. OUP Academic. Disponible en: \url{https://academic.oup.com/ptj/article-abstract/68/9/1371/2728422?redirectedFrom=fulltext} [Último acceso 25 de agosto, 2020].

\bibitem{velocidad_FES} Winchester, P., Carollo, J.J. \&amp; Habasevich, R., Physiologic costs of reciprocal gait in FES assisted walking. Nature News. Disponible en: \url{https://www.nature.com/articles/sc1994110} [Último acceso 25 de agosto, 2020].

\bibitem{velocidad_exoesqueleto_1} Yang A;Asselin P;Knezevic S;Kornfeld S;Spungen AM; Assessment of In-Hospital Walking Velocity and Level of Assistance in a Powered Exoskeleton in Persons with Spinal Cord Injury. Topics in spinal cord injury rehabilitation. Disponible en: \url{https://pubmed.ncbi.nlm.nih.gov/26364279/} [Último acceso 25 de agosto, 2020].

\bibitem{velocidad_exoesqueleto_2} Asselin, Pierre \& Knezevic, Steven \& Kornfeld, Stephen \& Cirnigliaro, Christopher \& Agranova-Breyter, Irina \& Bauman, William \& Spungen, Ann. (2015). Heart rate and oxygen demand of powered exoskeleton-assisted walking in persons with paraplegia. The Journal of Rehabilitation Research and Development. 52. 147-158. 10.1682/JRRD.2014.02.0060. Disponible en: \url{https://www.scopus.com/record/display.uri?eid=2-s2.0-84930973622&origin=inward} [Último acceso 25 de agosto, 2020].

\bibitem{distancia_exoesqueleto} Miller, L.E., Zimmermann, A.K. \&amp; Herbert, W.G., 2016. Clinical effectiveness and safety of powered exoskeleton-assisted walk: MDER. Medical Devices: Evidence and Research. Disponible en: \url{https://www.dovepress.com/clinical-effectiveness-and-safety-of-powered-exoskeleton-assisted-walk-peer-reviewed-article-MDER} [Último acceso 25 de agosto, 2020].

%sensores
\bibitem{sensores} Veltink, Peter \& Sinkjaer, Thomas \& Baten, Chris \& Bergveld, P. \& Spek, J. \& Haugland, Morten. (1998). Artificial and natural sensors in FES-assisted human movement control. 5. 2247 - 2250 vol.5. 10.1109/IEMBS.1998.744684. Disponible en: \url{https://www.researchgate.net/publication/3787424_Artificial_and_natural_sensors_in_FES-assisted_human_movement_control} [Último acceso 6 de mayo, 2020].

\bibitem{EMG} Mayo clinic, Anon, 2019. Electromyography (EMG). Mayo Clinic. Disponible en: \url{https://www.mayoclinic.org/tests-procedures/emg/about/pac-20393913} [Último acceso 6 de mayo, 2020].

\bibitem{EMG2} Klein, C.S. et al., 2018. Editorial: Electromyography (EMG) Techniques for the Assessment and Rehabilitation of Motor Impairment Following Stroke. Frontiers in neurology.  Disponible en: \url{https://www.ncbi.nlm.nih.gov/pmc/articles/PMC6305436/} [Último acceso 6 de mayo, 2020].


%----------------------------CAPITULO 2------------------------%
%electroestimulador
\bibitem{modulo_bluetooth} Bluetooth SMD Module - RN-42 (v6.15). WRL-12574 - SparkFun Electronics. Disponible en: \url{https://www.sparkfun.com/products/12574} [Último acceso 8 de junio, 2020].

\bibitem{programador} AVRISP mkII. Disponible en: \url{https://www.microchip.com/DevelopmentTools/ProductDetails/PartNO/ATAVRISP2} [Último acceso 10 de septiembre, 2020].

\bibitem{coolterm} Roger Meier's Freeware. Disponible en: \url{https://freeware.the-meiers.org/} [Último acceso 15 de noviembre, 2020].


%goniometros
\bibitem{goniometros} Twin-Axis Goniometers for Dynamic Joint Movement Analysis. Disponible en: \url{https://biometricsltd.com/goniometer.htm} [Último acceso 11 de junio, 2020].

\bibitem{goniometros_setup} Communications, S., Biometrics Goniometers and Torsiometers. NexGen Ergonomics - Products - Biometrics - Goniometers and Torsiometers. Disponible en: \url{http://www.nexgenergo.com/ergonomics/biosensors.html} [Último acceso 11 de junio, 2020].

%----------------------------CAPITULO 3------------------------%
%goniometros
\bibitem{eagle} EAGLE: PCB Design And Electrical Schematic Software. Autodesk. Disponible en: \url{https://www.autodesk.com/products/eagle/overview?plc=F360} [Último acceso 14 de noviembre, 2020].

\bibitem{convertidor_5vdc} TEL 2-0511. TEL 2-0511 | Traco Power. Disponible en: \url{https://www.tracopower.com/int/model/tel-2-0511} [Último acceso 14 de noviembre, 2020].

\bibitem{convertidor_15vdc} TDR 3-0513SM. TDR 3-0513SM | Traco Power. Disponible en: \url{https://www.tracopower.com/int/model/tdr-3-0513sm} [Último acceso 14 de noviembre, 2020].

\bibitem{snapeda} SnapEDA. Disponible en: \url{https://www.snapeda.com/} [Último acceso 14 de noviembre, 2020].




%----------------------------CAPITULO 5------------------------%
\bibitem{espressif} Development Boards. Development Boards | Espressif Systems. Disponible en: \url{https://www.espressif.com/en/products/devkits} [Último acceso 21 de noviembre, 2020].

\bibitem{picokit} ESP32-PICO-KIT V4 / V4.1 Getting Started Guide. ESP. Disponible en: \url{https://docs.espressif.com/projects/esp-idf/en/latest/esp32/hw-reference/esp32/get-started-pico-kit.html} [Último acceso 21 de noviembre, 2020].

\bibitem{devkitc} ESP32-DevKitC V4 Getting Started Guide. ESP. Disponible en: \url{https://docs.espressif.com/projects/esp-idf/en/latest/esp32/hw-reference/esp32/get-started-devkitc.html} [Último acceso 21 de noviembre, 2020].

\bibitem{esp32_consumo_cpu} ESP32 Series Datasheet. Disponible en: \url{https://www.espressif.com/sites/default/files/documentation/esp32_datasheet_en.pdf} [Último acceso 21 de noviembre, 2020].

\bibitem{modulos_esp32} ESP32 Series of Modules. ESP32 Wi-Fi amp; Bluetooth Modules I Espressif. Disponible en: \url{https://www.espressif.com/en/products/modules/esp32} [Último acceso 21 de noviembre, 2020].

\bibitem{duracion_ciclo_marcha} Kawalec, J.S., 2017. Mechanical testing of foot and ankle implants. Mechanical Testing of Orthopaedic Implants. Disponible en: \url{https://www.sciencedirect.com/science/article/pii/B9780081002865000123} [Último acceso 22 de noviembre, 2020].

\bibitem{duracion_fases_ciclo_marcha} JR;, G., An overview of normal walking. Instructional course lectures. Disponible en: \url{https://pubmed.ncbi.nlm.nih.gov/2186116/} [Último acceso 22 de noviembre, 2020].

\end{thebibliography}
